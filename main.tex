\documentclass[twocolumn, a4j, 9pt, dvipdfmx]{jsarticle}

\usepackage[margin=0.72in, bmargin=1.4in, tmargin=1.3in]{geometry}
\usepackage{graphicx}
\usepackage{url}
\usepackage{physics}

\usepackage{authblk}
\usepackage[linesnumbered,ruled,vlined,noresetcount]{algorithm2e}
\DontPrintSemicolon
\SetKwProg{Fn}{function}{:}{end}

\usepackage{color}
\newcommand{\red}[1]{\textcolor{red}{#1}}
\usepackage{amsmath,amssymb,amsthm}
\usepackage{physics}
\usepackage{float}
\newcommand{\vbm}[1]{\begin{bmatrix}#1\end{bmatrix}}  % 角括弧
\newcommand{\vpm}[1]{\begin{pmatrix}#1\end{pmatrix}}  % 丸括弧

\title{{\red(題目をここに書く)}}
\date{}
\author{山田太郎}

\setlength{\columnsep}{1cm}
\begin{document}

\maketitle

\section{目的}
\section{実験原理}
\section{実験方法}
\section{結果}
\section{考察}
引用は\cite{textbook}のように行う。
\section{結論}

% 図:画像はimg/に設置、文書中で\ref{fig:example}として引用できる。
% \begin{figure}[H]
%     \centering
%     \includegraphics[width=0.5\textwidth]{img/sample.png}
%     \caption{図の説明文}
%     \label{fig:example}
% \end{figure}

% 表: - の部分を数字に置き換えればOK。
% \begin{table}[H]
% \caption{表の説明文}
% \label{tab:example}
% \centering
% \begin{tabular}{l|llll}\hline\hline
% & - () & - () & - () & - () \\ \hline
% - & - & - & - & - \\
% - & - & - & - & - \\
% - & - & - & - & - \\ \hline
% \end{tabular}
% \end{table}

% 方程式or等式
% \begin{equation}
% E\{\epsilon^2\} = E\{(x(t) - y(t - \tau))^2\}
% \label{eq:example}
% \end{equation}

% 式変形
% \begin{align}
% E\{\epsilon^2\} &= E\{x(t)^2 - 2x(t)y(t - \tau)) + y(t - \tau)^2\} \notag \\
% &= E\{x(t)^2\} + E\{y(t - \tau)^2\} - 2E\{x(t)y(t - \tau)\}
% \label{eq:example}
% \end{align}

\bibliography{Name} %''Name.bib''という名前のスタイルファイルをtexファイルと同一フォルダに配置してbibtexでコンパイル
\bibliographystyle{abbrv} 
\end{document}